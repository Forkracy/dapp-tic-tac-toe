\subsection{Check for Winner}\label{subsec:check}
\begin{lstlisting}[caption={Check for Winner Function on the Smart Contract}, label={lst:check},language=JavaScript,escapechar=|]
function checkForWinner(uint x, uint y, uint gameId, address currentPlayer) private {
	Game storage game = games[gameId];
	
	//is winning already possible?
	if (game.moveCounter < 2 * boardSize - 1) |\label{line2:win}|
		return;
	}
	
	SquareState symbol = game.board[y][x];
	
	//check column
	for (uint i = 0; i < boardSize; i++) { |\label{line2:check1}|
		if (game.board[i][x] != symbol) {
			break;
		}
		else if (i == (boardSize - 1)) {
			game.winnerAddr = currentPlayer;
			game.state = getGameState(symbol);
			payoutBets(game.gameId); |\label{line2:pay1}|
			return;
		}
	}
	
	//check row
	for (i = 0; i < boardSize; i++) {	|\label{line2:check2}|
		if (game.board[y][i] != symbol) {
			break;
		}
		else if (i == (boardSize - 1)) {
			game.winnerAddr = currentPlayer;
			game.state = getGameState(symbol);
			payoutBets(game.gameId); |\label{line2:pay2}|
			return;
		}
	}
	
	//check diagonal: (x-y) 0-0, 1-1, 2-2 
	if (x == y) {					|\label{line2:check3}|
		for (i = 0; i < boardSize; i++) {
			if (game.board[i][i] != symbol) {
				break;
			}
			else if (i == (boardSize - 1)) {
				game.winnerAddr = currentPlayer;
				game.state = getGameState(symbol);
				payoutBets(game.gameId); |\label{line2:pay3}|
				return;
			}
		}
	}
	
	// check antidiagonal: (x-y) 2-0, 1-1, 0-2
	if (x + y == (boardSize - 1)) {			|\label{line2:check4}|
		for (i = 0; i < boardSize; i++) {
			if (game.board[i][boardSize - 1 - i] != symbol) {
				break;
			}
			else if (i == (boardSize - 1)) {
				game.winnerAddr = currentPlayer;
				game.state = getGameState(symbol);
				payoutBets(game.gameId);|\label{line2:pay4}|
				return;
			}
		}
	}
	//check for draw
	if (game.moveCounter == boardSize * boardSize) { |\label{line2:check5}|
		game.state = GameState.DRAW;
		payoutBets(game.gameId); |\label{line2:pay5}|
	}
}
\end{lstlisting}

The logic for the evaluating the winner function is shown in Listing \ref{lst:check}. The function stop if the number of moves done are not enough to possibly have a winner (line \ref{line2:win}). After it goes through all possible combination of winning and checks if there are the same symbol a line (line \ref{line2:check1}, \ref{line2:check2}, \ref{line2:check3} and \ref{line2:check4}). If it does not found any winning row it checks at last for draw (line \ref{line2:check5}). Is there a winning line found or the game is a draw, the payout function gets called (line \ref{line2:pay1}, \ref{line2:pay2}, \ref{line2:pay3}, \ref{line2:pay4} and \ref{line2:pay5}).